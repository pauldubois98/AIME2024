Achieving optimal dose distribution in radiation therapy planning is a complex task, with contradicting goals.
Yet, this step is crucial with profound implications for patient treatment.

The absence of universally agreed-upon constraints prioritization in radiation therapy planning complicates the definition of an optimal plan, requiring a delicate balance between multiple objectives.
This balance usually ends up being done manually.

The optimization process is further hindered by complex mathematical aspects, involving non-convex multi-objective inverse problems with a vast solution space.
Expert bias introduces variability in clinical practice, as treatment planning is shaped by the preferences and expertise of radiation oncologists and medical physicists.

To surmount these challenges, we propose a first step towards a fully automated approach, using an innovative deep-learning method that allows automatic navigation toward acceptable solution.
We successfully trained an agent evaluating actions of a human dosimetrist, in order to reach a plan similar to past cases.
As this is very new and ongoing research, we generated synthetic phantom patients and associated trust-able clinical doses.
In future work, we hope to be able to apply this technique to real cases.