\section{Introduction}

Radiation therapy is now a reliable treatment for oncology.
Despite this consensus, the way to deliver radiotherapy for its best result remain very dependent upon doctors.
Moreover, it appears that there is a large variability across centers.

To achieve the best treatment, doctors need to solve a complex inverse mathematical optimization problem with multiples trade-offs.
There is a lack of standardized prioritization of constraints make the optimization a real challenge.
The standard procedure nowadays is to manually guide computer optimization: dosimetrists manually update the settings of an optimizing software (so called Treatment Planning System).

There has been many tries to create a metric that quantify the quality of a treatment plan: Normal tissue complication probabilities (NTCP), Target coverage, Conformity index, Heterogeneity index...
However, none of them have been able to satisfy all radio-oncologists, and the only reliable way of assessing a plan for doctors is to check out the dose-volume histograms (DVHs) them-self.

As this is very new and ongoing research, we generated synthetic phantom patients and associated trust-able clinical dose.
In future work, we hope to be able to apply this technique to real cases.

\section{Materials and Methods}

\subsection{Reinforcement Learning Reward}

\subsection{Reward-Free Reinforcement Learning}


\section{Results}

\section{Discussion}



\section*{Appendix}

\subsection*{Synthetic phantom patients}

\subsection*{Clinical dose}

\subsection*{Optimization}

\subsection*{Evaluation}
