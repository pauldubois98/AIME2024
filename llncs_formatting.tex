% This is samplepaper.tex, a sample chapter demonstrating the
% LLNCS macro package for Springer Computer Science proceedings;
% Version 2.20 of 2017/10/04
%
\documentclass[runningheads]{llncs}
%
\usepackage{graphicx}
% Used for displaying a sample figure. If possible, figure files should
% be included in EPS format.
%
% If you use the hyperref package, please uncomment the following line
% to display URLs in blue roman font according to Springer's eBook style:
% \renewcommand\UrlFont{\color{blue}\rmfamily}

\begin{document}
	\title{Radiotherapy Dose Optimization via Clinical Knowledge Based Reinforcement Learning}
	\titlerunning{Radiotherapy Optimization with Clinical Knowledge}
	
	\author{Paul Dubois\inst{1,2}\orcidID{0009-0003-3856-8048}
		\and \\
		Paul-Henry Cournède \inst{1}\orcidID{0000-0001-7679-6197}
		\and \\
		Nikos Paragios\inst{2}\orcidID{0000-0002-9668-4763}
%		\and \\
%		Rafael Marini-Silva\inst{2}
%		\and \\
%		Norbert Bus\inst{2}
%		\and \\
%		Nikos Komodakis\inst{2}
%		\and \\
%		Pascal Fenoglietto\inst{3}
	}
	% \authorrunning{F. Author et al.}
	% First names are abbreviated in the running head.
	% If there are more than two authors, 'et al.' is used.
	%
	\institute{Biomathematics, MICS, CentraleSupélec, Université Paris-Saclay
		\email{\{p.dubois,paul-henry.cournede\}@centralesupelec.fr}\\
		\url{https://biomathematics.mics.centralesupelec.fr/}
		\and
		TheraPanacea, Paris, France\\
		\email{\{p.dubois,n.paragios\}@therapanacea.eu}\\
%		\email{\{p.dubois,n.paragios,r.marini-silva,n.bus,n.komodakis\}@therapanacea.eu}\\
		\url{https://www.therapanacea.eu/}
%		\and
%		Institut du Cancer de Montpellier (Val d'Aurelle), Montpellier, France\\
%		\email{paul.dubois@icm.unicancer.fr}\\
%		\email{\{paul.dubois,pascal.fenoglietto\}@therapanacea.eu}\\
%		\url{https://www.therapanacea.eu/}
	}
	
	\maketitle
	
	\begin{abstract}
%		The abstract should briefly summarize the contents of the paper in 15--250 words.
		% Achieving optimal dose distribution in radiation therapy planning is a complex task, with contradicting goals. Yet, this step is crucial with profound implications for patient treatment. % delete?

The absence of universally agreed-upon constraints prioritization in radiation therapy planning complicates the definition of an optimal plan, requiring a delicate balance between multiple objectives. This balance usually ends up being done manually. %merge to 1 sentence

The optimization process is further hindered by complex mathematical aspects, involving non-convex multi-objective inverse problems with a vast solution space.
Expert bias introduces variability in clinical practice, as treatment planning is shaped by the preferences and expertise of radiation oncologists and medical physicists.

To surmount these challenges, we propose a first step towards a fully automated approach, using an innovative deep-learning method.

As this is new/never-performed/ongoing research, we generated synthetic phantom patients and associated trustworthy/realistic clinical doses.% need to be up in the abstract

This method allows automatic navigation toward acceptable solution via [blah].
% sentece abour method here
We successfully trained an agent evaluating actions of a human dosimetrist, in order to reach a plan similar to past cases.
% By evaluating the action taken by humans dosimetrists, we succesfully trained an agent capable of optimizing a plan similar to past cases

In future work, we hope to be able to apply this technique to real cases.

% 150-300 words

% too much background
% 1§
% 6-8 sentences
% more details on the method (2 sentences)
% tie ccl to reason why
% last sentence is ok

% one sentence per § of bg


		\keywords{Radiotherapy \and Dose Optimization \and Reinforcement Learning \and Deep Learning.}
	\end{abstract}
	
	% the environments 'definition', 'lemma', 'proposition', 'corollary',
	% 'remark', and 'example' are defined in the LLNCS documentclass as well.
	\section{Introduction}
In contemporary radiation therapy, photon intensity modulated radiation therapy (IMRT) stands as a pivotal technique utilized to attain precise and conformal dose distributions within target volumes[add citation].
This achievement owes its realization chiefly to the advent of the multileaf collimator (MLC)[add citation].

Radiation therapy is now a reliable treatment for oncology [add citation].
Despite this consensus, the way to deliver radiotherapy for its best result remain very dependent upon doctors.
Moreover, it appears that there is a large variability across centers[add citation?].

To achieve the best treatment, doctors need to solve a complex inverse mathematical optimization problem with multiples trade-offs[add citation].
There is a lack of standardized prioritization of constraints make the optimization a real challenge[add citation].
The standard procedure nowadays is to manually guide computer optimization: dosimetrists manually update the settings of an optimizing software (so called Treatment Planning System)[add citation].

There has been many tries to create a metric that quantify the quality of a treatment plan: Normal tissue complication probabilities (NTCP), Target coverage, Conformity index, Heterogeneity index (non-exhaustive list)[add citations].
However, none of them has been able to satisfy all radio-oncologists, and the only reliable way of assessing a plan for doctors is to check out the dose-volume histograms (DVHs) them-self.

As a result, Pareto surface exploration are doomed to failure due to the lack of impartial quantitative measurement for a particular plan[add citation].
Other meta-optimization techniques are similarly bounded, for the same reason[add citation].
An extra challenge to attend for those is the fact that not all cases have the same "difficulty".
Hence, for an "easy" case, doctors will require an excellent dose (in terms of the metrics mentioned above), while they can be more permissive for "harder" cases.
This make the acceptability of a plan hard to define in general.

As this is very new and ongoing research, we generated synthetic phantom patients and associated trust-able clinical dose.
In future work, we hope to be able to apply this technique to real cases.

\section{Materials and Methods}

\subsection{Reinforcement Learning Reward}

\subsection{Reward-Free Reinforcement Learning}


\section{Results}

\section{Discussion}



\section*{Appendix}

\subsection*{Synthetic phantom patients}

\subsection*{Clinical dose}

\subsection*{Optimization}

\subsection*{Evaluation}

	
	% ---- Bibliography ----
	% BibTeX users should specify bibliography style 'splncs04'.
	% References will then be sorted and formatted in the correct style.
	\bibliographystyle{splncs04}
	\bibliography{refs.bib}
\end{document}