% This is samplepaper.tex, a sample chapter demonstrating the
% LLNCS macro package for Springer Computer Science proceedings;
% Version 2.20 of 2017/10/04
%
\documentclass[runningheads]{llncs}
%
\usepackage{graphicx}
% Used for displaying a sample figure. If possible, figure files should
% be included in EPS format.
%
% If you use the hyperref package, please uncomment the following line
% to display URLs in blue roman font according to Springer's eBook style:
% \renewcommand\UrlFont{\color{blue}\rmfamily}

\begin{document}
	\title{Radiotherapy Dose Optimization via Clinical Knowledge Based Reinforcement Learning}
	\titlerunning{Radiotherapy Optimization with Clinical Knowledge}
	
	\author{Paul Dubois\inst{1,2}\orcidID{0009-0003-3856-8048}
		\and \\
		Paul-Henry Cournède \inst{1}\orcidID{0000-0001-7679-6197}
		\and \\
		Nikos Paragios\inst{2}\orcidID{0000-0002-9668-4763}
%		\and \\
%		Rafael Marini-Silva\inst{2}
%		\and \\
%		Norbert Bus\inst{2}
%		\and \\
%		Nikos Komodakis\inst{2}
		\and \\
		Pascal Fenoglietto\inst{3}
	}
	% \authorrunning{F. Author et al.}
	% First names are abbreviated in the running head.
	% If there are more than two authors, 'et al.' is used.
	%
	\institute{Biomathematics, MICS, CentraleSupélec, Université Paris-Saclay
		\email{\{p.dubois,paul-henry.cournede\}@centralesupelec.fr}\\
		\url{https://biomathematics.mics.centralesupelec.fr/}
		\and
		TheraPanacea, Paris, France\\
		\email{\{p.dubois,n.paragios\}@therapanacea.eu}\\
%		\email{\{p.dubois,n.paragios,r.marini-silva,n.bus,n.komodakis\}@therapanacea.eu}\\
		\url{https://www.therapanacea.eu/}
		\and
		Institut du Cancer de Montpellier (Val d'Aurelle), Montpellier, France\\
		\email{paul.dubois@icm.unicancer.fr}\\
		\email{\{paul.dubois,pascal.fenoglietto\}@therapanacea.eu}\\
		\url{https://www.therapanacea.eu/}
	}
	
	\maketitle
	
	\begin{abstract}
%		The abstract should briefly summarize the contents of the paper in 15--250 words.
		% Achieving optimal dose distribution in radiation therapy planning is a complex task, with contradicting goals. Yet, this step is crucial with profound implications for patient treatment.
% The absence of universally agreed-upon constraints prioritization in radiation therapy planning complicates the definition of an optimal plan, requiring a delicate balance between multiple objectives. This balance usually ends up being done manually.
A radiation therapy plan finds an equilibrium between goals with no universal prioritization. The delicate balance between multiple objectives is typically done manually.
The optimization process is further hindered by complex mathematical aspects, involving non-convex multi-objective inverse problems with a vast solution space.
Expert bias introduces variability in clinical practice, as the preferences of radiation oncologists and medical physicists shape treatment planning.

To surmount these challenges, we propose a first step towards a fully automated approach, using an innovative deep-learning framework.
Using a clinically meaningful distance between doses, we trained a reinforcement learning agent to mimic a set of plans.
This method allows automatic navigation toward acceptable solutions via the exploitation of optimal dose distributions found by human planners on previously treated patients.
As this is ongoing research, we generated synthetic phantom patients and associated realistic clinical doses.
Our deep learning agent successfully learned correct actions leading to treatment plans similar to past cases ones.
The incapacity to reproduce human-like dose plans hinders adopting a fully automated treatment planning system; this method could start paving the way towards human-less treatment planning system technologies. % "tie conclusion of reasoning why we do this technique."
In future work, we hope to be able to apply this technique to real cases.

% 150-300 words => 200
% 6-8 sentences => 11 (but short)


		\keywords{Radiotherapy \and Dose Optimization \and Reinforcement Learning \and Deep Learning.}
	\end{abstract}
	
	% the environments 'definition', 'lemma', 'proposition', 'corollary',
	% 'remark', and 'example' are defined in the LLNCS documentclass as well.
	\section{Introduction}

Radiation therapy is now a reliable treatment for oncology.
Despite this consensus, the way to deliver radiotherapy for its best result remain very dependent upon doctors.
Moreover, it appears that there is a large variability across centers.

To achieve the best treatment, doctors need to solve a complex inverse mathematical optimization problem with multiples trade-offs.
There is a lack of standardized prioritization of constraints make the optimization a real challenge.
The standard procedure nowadays is to manually guide computer optimization: dosimetrists manually update the settings of an optimizing software (so called Treatment Planning System).


As this is very new and ongoing research, we generated synthetic phantom patients and associated trust-able clinical dose.
In future work, we hope to be able to apply this technique to real cases.

\section{Materials and Methods}

This is how you add one \cite{Pivot2023} or multiple citations \cite{Saporta2022,Robert2022}.

% example equation
And this is a dummy equation
\begin{equation}
I_\alpha = \int_0^\alpha f(x) dx
\end{equation}

% example two column figure (use figure instead of figure* for one column figure)
\begin{figure*}
  \centering
  \includegraphics[width=1.0\textwidth]{./fig1.pdf}
  \caption{This a dummy figure to be replaced.}
  \label{fig:dummyfigure}
\end{figure*}

Figure \ref{fig:dummyfigure} shows how to include a figure.
Table \ref{tab:dummytable} shows how to include a table.

\bigskip

\lipsum[2-7]

\section{Results}
\subsection{Simulation results}

\lipsum[2-3]

\subsection{Other results}
\lipsum[2-3]

% example table
\begin{table}
  \centering
  \begin{tabular}{lrrr}
  \toprule
  $\alpha$   & $\beta$ & $\gamma$ & $\delta$ \\
  \midrule
  A          & 1       & a        & 3        \\
  B          & 2       & b        & 2        \\
  C          & 3       & c        & 1        \\
  \bottomrule
  \end{tabular}
  \caption{This is a dummy table to be replaced.}
  \label{tab:dummytable}
\end{table}



\section{Discussion}
\lipsum[2-9]


\section{Conclusion}
\lipsum[2]



	
	% ---- Bibliography ----
	% BibTeX users should specify bibliography style 'splncs04'.
	% References will then be sorted and formatted in the correct style.
	\bibliographystyle{splncs04}
	\bibliography{refs.bib}
\end{document}