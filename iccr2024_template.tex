% Latex template for submission to the XXth International Conference on the use of Computers in Radiation therapy
% (ICCR 2024)
%
% Date:   Oct 2023
%
% Modified from
% https://github.com/gschramm/fully3d_templates
% 
% To build this document, we recommend to use latexmk via:
% ```latexmk -pdf iccr2024_template.tex```
% Building in the online editor overleaf also works.

\documentclass[11pt,twocolumn,twoside]{article}
\usepackage{iccr2024}

%%%%%% add your extra packages here (if needed)                                        %%%%%
%%%%%% before, have a look which packages are already imported by the iccr2024 package %%%%%
%\usepackage{mypackage}


%%%%% add your bibtex file that contains the bibtex entries here %%%%%
%%%%% please include DOIs in the bibtex entries if possible      %%%%%
\addbibresource{iccr2024_template.bib}

\begin{document}


%-------------------------------------------------------------------------------------------
%%%%% add your title here %%%%%
\title{Deep Parameters Evaluation for Automatic Treatment Planning System} 

%%%%% add authors and affiliations here %%%%%
\author[1,2]{Paul Dubois}
\author[1]{Nikos Paragios}
\author[2]{Paul-Henry Cournède}
%\author[1]{Rafael Marini-Silva}
%\author[1]{Norbert Bus}
%\author[3]{Nikos Komodakis}

\affil[1]{TheraPanacea, Paris, France}

\affil[2]{MICS, CentraleSupélec, Université Paris-Saclay, Paris, France}

%%%%% don't change these 2 lines %%%%%
\maketitle
\thispagestyle{fancy}



%-------------------------------------------------------------------------------------------
%%%%% add your summary (abstract) here               %%%%%%
%%%%% use footnotesize for this section              %%%%%%
%%%%% please stick to the customabstract environment %%%%%% 


\begin{customabstract}
	Achieving optimal dose distribution in radiation therapy planning is a complex task, with contradicting goals.
	Yet, this step is crucial with profound implications for patient treatment and toxicity management.
	
	The absence of universally agreed-upon constraints prioritization in radiation therapy planning complicates the definition of an optimal plan, requiring a delicate balance between multiple objectives.
	This balanced usually ends up being done manually.
	
	The optimization process is further hindered by complex mathematical aspects, involving non-convex multi-objective inverse problems with a vast solution space.
	Expert bias introduces variability in clinical practice, as treatment planning is shaped by the preferences and expertise of radiation oncologists and medical physicists.
	
	To surmount these challenges, we propose a first step towards a fully automated approach, using an innovative deep learning method that allows automatic navigation towards acceptable solution.
	We successfully trained an agent evaluating actions of a human dosimetrist, in order to reach a plan similar to past history.
	As this is very new and ongoing research, we generated synthetic phantom patients and associated trust-able clinical dose.
	In future work, we hope to be able to apply this technique to real cases.
\end{customabstract}


%-------------------------------------------------------------------------------------------
%%%%% main text                                                %%%%%    
%%%%% remove the dummy content and put your own content here   %%%%% 
%%%%% feel free to choose your own section titles              %%%%% 
%%%%% you don't need to put the content in a separate tex file %%%%%

% dummy_content.tex shows how to add sections, figures, tables, formulas, and references
% remove the following line, it just adds dummy content
\section{Introduction}
In contemporary radiation therapy, photon intensity modulated radiation therapy (IMRT) stands as a pivotal technique utilized to attain precise and conformal dose distributions within target volumes[add citation].
This achievement owes its realization chiefly to the advent of the multileaf collimator (MLC)[add citation].

Radiation therapy is now a reliable treatment for oncology [add citation].
Despite this consensus, the way to deliver radiotherapy for its best result remain very dependent upon doctors.
Moreover, it appears that there is a large variability across centers[add citation?].

To achieve the best treatment, doctors need to solve a complex inverse mathematical optimization problem with multiples trade-offs[add citation].
There is a lack of standardized prioritization of constraints make the optimization a real challenge[add citation].
The standard procedure nowadays is to manually guide computer optimization: dosimetrists manually update the settings of an optimizing software (so called Treatment Planning System)[add citation].

There has been many tries to create a metric that quantify the quality of a treatment plan: Normal tissue complication probabilities (NTCP), Target coverage, Conformity index, Heterogeneity index (non-exhaustive list)[add citations].
However, none of them has been able to satisfy all radio-oncologists, and the only reliable way of assessing a plan for doctors is to check out the dose-volume histograms (DVHs) them-self.

As a result, Pareto surface exploration are doomed to failure due to the lack of impartial quantitative measurement for a particular plan[add citation].
Other meta-optimization techniques are similarly bounded, for the same reason[add citation].
An extra challenge to attend for those is the fact that not all cases have the same "difficulty".
Hence, for an "easy" case, doctors will require an excellent dose (in terms of the metrics mentioned above), while they can be more permissive for "harder" cases.
This make the acceptability of a plan hard to define in general.

As this is very new and ongoing research, we generated synthetic phantom patients and associated trust-able clinical dose.
In future work, we hope to be able to apply this technique to real cases.

\section{Materials and Methods}

\subsection{Reinforcement Learning Reward}

\subsection{Reward-Free Reinforcement Learning}


\section{Results}

\section{Discussion}



\section*{Appendix}

\subsection*{Synthetic phantom patients}

\subsection*{Clinical dose}

\subsection*{Optimization}

\subsection*{Evaluation}



%-------------------------------------------------------------------------------------------
\printbibliography

\end{document}
