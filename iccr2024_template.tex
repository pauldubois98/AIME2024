% Latex template for submission to the XXth International Conference on the use of Computers in Radiation therapy
% (ICCR 2024)
%
% Date:   Oct 2023
%
% Modified from
% https://github.com/gschramm/fully3d_templates
% 
% To build this document, we recommend to use latexmk via:
% ```latexmk -pdf iccr2024_template.tex```
% Building in the online editor overleaf also works.

\documentclass[11pt,twocolumn,twoside]{article}
\usepackage{iccr2024}

%%%%%% add your extra packages here (if needed)                                        %%%%%
%%%%%% before, have a look which packages are already imported by the iccr2024 package %%%%%
%\usepackage{mypackage}


%%%%% add your bibtex file that contains the bibtex entries here %%%%%
%%%%% please include DOIs in the bibtex entries if possible      %%%%%
\addbibresource{iccr2024_template.bib}

\begin{document}


%-------------------------------------------------------------------------------------------
%%%%% add your title here %%%%%
\title{A LaTex Template for a ICCR 2024 submission} 

%%%%% add authors and affiliations here %%%%%
\author[1,2]{Paul Dubois}
\author[1]{Nikos Paragios}
\author[2]{Paul-Henry Cournède}
%\author[1]{Rafael Marini-Silva}
%\author[1]{Norbert Bus}
%\author[3]{Nikos Komodakis}

\affil[1]{TheraPanacea, Paris, France}

\affil[2]{MICS, CentraleSupélec, Université Paris-Saclay, Paris, France}

%%%%% don't change these 2 lines %%%%%
\maketitle
\thispagestyle{fancy}



%-------------------------------------------------------------------------------------------
%%%%% add your summary (abstract) here               %%%%%%
%%%%% use footnotesize for this section              %%%%%%
%%%%% please stick to the customabstract environment %%%%%% 


\begin{customabstract}
	Achieving optimal dose distribution in radiation therapy planning is a complex task, with contradicting goals.
	Yet, this step is crucial with profound implications for patient treatment and toxicity management.
	
	The absence of universally agreed-upon constraints prioritization in radiation therapy planning complicates the definition of an optimal plan, requiring a delicate balance between multiple objectives.
	This balanced usually ends up being done manually.
	
	The optimization process is further hindered by complex mathematical aspects, involving non-convex multi-objective inverse problems with a vast solution space.
	Expert bias introduces variability in clinical practice, as treatment planning is shaped by the preferences and expertise of radiation oncologists and medical physicists.
	
	To surmount these challenges, we propose a first step towards a fully automated approach, using an innovative deep reinforcement learning like method.
	We managed to successfully train an agent to mimic the actions of a human dosimetrist, in order to reach a plan similar to past history.
	As this is very new and ongoing research, we generated synthetic phantom patients and associated trust-able clinical dose.
	In future work, we hope to be able to apply this technique to real cases.
\end{customabstract}


%-------------------------------------------------------------------------------------------
%%%%% main text                                                %%%%%    
%%%%% remove the dummy content and put your own content here   %%%%% 
%%%%% feel free to choose your own section titles              %%%%% 
%%%%% you don't need to put the content in a separate tex file %%%%%

% dummy_content.tex shows how to add sections, figures, tables, formulas, and references
% remove the following line, it just adds dummy content
\input{dummy_content.tex}


%-------------------------------------------------------------------------------------------
\printbibliography

\end{document}
